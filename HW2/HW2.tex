\documentclass[11pt]{article}
\usepackage{../EllioStyle}

\title{Homework 1}
\author{Elliott Pryor}
\date{31 Jan 2021}

\rhead{Homework 1}

\begin{document}
\maketitle

\problem{1} 6.2.4 Problem 6

Prove that if $f$ is Riemann integrable on $[a, b]$ and $g(x) = f(x)$ for
every $x$ except for a finite number, then $g$ is Riemann integrable.

\textbf{Hint:} Mimic the proof of Theorem 6.2.3

\hrule

\begin{proof}
    
    Let $a_1, a_2, ..., a_N$ denote the points where $g(a_i) \neq f(a_i)$.
    Given any $1/n$ surround each $a_k$ by an interval $I_k$ such that $|I_k| < 1/n$. Then,
    $g$ is continuous and Riemann integrable on $[a,b]$ with $\cup_{k = 1} ^N I_k$ removed.

    Then, we estimate the contribution to the oscillation from the $I_k$ intervals. First, $f$ must be bounded since
    it is Riemann-integrable. Then since $g$ differs from $f$ at a finite number of locations, it must also be bounded. 
    Then let $M = \sup _{x \in [a,b]} g(x)$ and $M = \inf _{x \in [a,b]} g(x)$. Then the contribution of any of the intervals:
    $I_k$ is at most $1/n \cdot (M-m)$. There are $N$ such intervals, so the total contribution to oscillation is $N/n \cdot (M - m)$.

    Then for the remaining intervals: $[a,b] \setminus \cup_{k = 1} ^N I_k$, $g = f$. Since $f$ is Riemann-integrable,
    the oscillation on these intervals can be made sufficiently small by choosing the partition size sufficiently small. 
    In other words, the total oscillation on a partition $P'$ of $[a,b] \setminus \cup_{k = 1} ^N I_k$, $g = f$ can be,
    $\forall 1/n \sp \exists 1/m \sp st \sp Osc(g, P') < 1/n$ for $|P'| < 1/m$ (part b of Theorem 6.2.1)

    Then the total oscillation $Osc(g, P) < 1/n + N/n \cdot (M - m)$. Since $N, M, m$ are constant, there exists a sequence of 
    partitions such that $Osc(g, P_j) \to 0 \sp as \sp j \to \infty$ by selecting $n$ large enough.


\end{proof}



\problem{2} 6.2.4 Problem 9, part b

If $f$ is Riemann integrable on $[a, b]$, prove it satisfies a Lipschitz condition

\hrule




\problem{3} 6.2.4 Problem 10

If $f$ is Riemann integrable on $[a, b]$ and continuous at $x_0$, prove
that $F(x) = \int_a ^x f(t) dt$ is differentiable at $x_0$ and $F'(x_0) = f(x_0)$.
Show that if $f$ has a jump discontinuity at $x_0$, then $F$ is not
differentiable at $x_0$. 

\textbf{Hint: } Refer to the proof of Theorem 6.1.2, note that $f(x)$ being continuous at $x_0$
can be written as $\forall 1/m, \sp \exists 1/n \sp st. \sp \forall x \in [a,b], \sp |x - x_0| < 1/n$, we have $|f(x) - f(x_0)| < 1/m$
or $f(x_0) - 1/m < f(x) < f(x_0) + 1/m$
\hrule




\end{document}