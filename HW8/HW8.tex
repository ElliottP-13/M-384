\documentclass[11pt]{article}
\usepackage{../EllioStyle}

\title{Homework 8}
\author{Elliott Pryor}
\date{9 April 2021}

\rhead{Homework 9}

\begin{document}
\maketitle

\problem{1} 10.1.5 Problem 3

If $f$ is differentiable at $y$, show that $d_uf(y)$ is linear in $u$,
meaning $d_{(au + bv)}f(y) = a d_u(y) + b d_v f(y)$.

\textbf{Hint:} Apply Theorem 10.1.1

\hrule


\begin{proof}
    
    We know from theorem 10.1.1, since $f$ is differentiable at $y$, then
    $d_{(au + bv)}f(y) = df(y) (au + bv) = a \cdot df(y) u + b \cdot df(y)  v$
    since matrix multiplication is distributive. 
    We then recognize that $df(y) u$ has the form (from theorem 10.1.1) of $d_u(y)$,
    and similarly for $df(y) v$. So we have 
    $d_{(au + bv)}f(y) = df(y) (au + bv) = a \cdot df(y) u + b \cdot df(y)  v = a d_u(y) + b d_v f(y)$
\end{proof}





\problem{2} 10.1.5 Problem 10

Let $g : [a,b] \to \reals^n$ be differentiable. If $f : \reals^n \to \reals$ is differentiable,
what is the derivative $(d/dt)f(g(t))$

\textbf{Hint:} Use notation $g(t) = (g_1(t), \dots, g_n(t)), t \in [a,b]$
and $f(z) = f(z_1, \dots, z_n), z = (z_1, \dots z_n) \in \reals^n$ and Apply
the chain rule.

\hrule


\begin{proof}
    
    we know the Chain rule in general is: 
    $$\frac{\partial f}{\partial x_j} = \sum_{k = 1} ^n \frac{\partial f}{\partial z_k} \frac{\partial z_k}{\partial x_j}$$
    where $z_k = g_k(x_1, \dots x_n)$.

    In our case, we are looking for $x_j = t$, and $g(t) = (g_1(t), \dots, g_n(t)), t \in [a,b]$
    So $z_k = g_k(t)$.

    So we have 
    $$\frac{\partial f}{\partial t} = \sum_{k = 1} ^n \frac{\partial f}{\partial z_k} \frac{\partial g_k(t)}{\partial t}$$
    $\frac{\partial g_k(t)}{\partial t}$ is just a number (scalar), so we can write this as two separate sums:
    $\sum_{k = 1} ^n \frac{\partial f}{\partial z_k} + \sum_{k = 1} ^n \frac{\partial g_k(t)}{\partial t}$.
    Then clearly this is just the sum of all partial derivatives of each function, which is the differential $df, dg$
    So we have:

    $$\frac{\partial f}{\partial t} = df \, dg$$

\end{proof}






\problem{3} 10.1.5 Problem 13

Compute $df$ of 
\begin{enumerate}
    \item $f: \reals^2 \to \reals, \sp f(x_1, x_2) = x_1 e^{x_2}$
    \item $f: \reals^3 \to \reals, \sp f(x_1, x_2, x_3) = (x_3, x_2)$
    \item $f: \reals^2 \to \reals^3, \sp f(x_1, x_2) = (x_1, x_2, x_1\cdot x_2)$
\end{enumerate}

\hrule







\problem{4} 10.1.5 Problem 15

If $f: D \to \reals$ is $C^1$ with $D \subseteq \reals^n$ and $D$ contains the line segment
joining $x$ and $y$, show that $f(y) = f(x) +  \nabla(z) \cdot (y - x)$ for some
point $z$ on the line segment. Explain why this is an $n$-dimensional
analog of the mean value theorem


\textbf{Hint:} Define function $g : [0, 1] \to \reals^n$ by $g(t) = x + t(y - x)$
and consider the composition function 
$$h(t) = (f \circ g)(t) = f(g(t)) : \reals \to \reals$$
Apply Mean Value Theorem to $h(t)$ for $h(1)- h(0)$ and use the chain rule (formula derived
in problem 10 above) to calculate $h'$
\hrule





\end{document}