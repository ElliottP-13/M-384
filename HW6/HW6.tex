\documentclass[11pt]{article}
\usepackage{../EllioStyle}

\title{Homework 6}
\author{Elliott Pryor}
\date{19 March 2021}

\rhead{Homework 5}

\begin{document}
\maketitle

\problem{1} 7.5.5 Problem 7

If $f$ is $C^1$ on $[a,b]$ prove that there exists a cubic polynomial $P$
such that $f- P$ and its first derivative vanish at the endpoints of the interval.

\textbf{Hint: } you can use the result of problem 1 without proving it

\hrule

\begin{proof}
    
    From problem 1 we know that there exists a polynomial $P$ of degree $2n -1$ satisfying
    $P(x_k) = a_k$ and $P'(x_k) = b_k$ for $k = 1... n$. In our case we want a cubic polynomial,
    so $n = 2$. Thus we have $P(x_1) = a_1$, $P(x_2) = a_2$ and $P'(x_k) = b_k$, $P'(x_2) = b_2$.
    Let $x_1 = a$ and $x_2 = b$. We then let $a_1 = f(a)$ and $a_2 = f(b)$, and similarly $b_1 = f'(a), b_2 = f'(b)$.

    Then by our construction of $P$ we have that $f(a) - P(a) = f(a) - a_1 = 0$, and $f(b) - P(b) = f(b) - a_2 = 0$.
    Satisfying the first condition. Then similarly, $f'(a) - P'(a) = f'(a) - b_1 = 0$ and $f'(b) - P'(b) = f'(b) - b_2 = 0$.
    And $P$ is of degree 3. Thus we have a cubic polynomial $P$
    such that $f- P$ and its first derivative vanish at the endpoints of the interval.

\end{proof}



\problem{2} 7.5.5 Problem 9 

If $f(c) = 0$ for some $c \in (a,b)$ prove that the polynomials approximating $f$ on 
$[a,b]$ may be taken to vanish at $c$.

\textbf{Hint: } Here $f(x)$ is a continuous function on $[a,b]$.
Assume $f_n(x)$ is the sequence of polynomials approximating $f(x)$ uniformly by WTA,
consider $g_n(x) = f_n(x) - f_n(c)$

\hrule


\begin{proof}
    
    Let $f_n$ be the sequence of polynomials approximating $f$ by the Wierstrass Approximation Theorem
    By WAT, we know that $f_n \to f$ uniformly. Thus
    $\forall 1/m \sp \exists N \sp st \sp \forall n \geq N \sp \forall x \in [a,b] \sp |f_n(x) - f(x)| \leq 1/m$.
    So we know that at $x = c$ we have $\forall 1/m \sp \exists N \sp st \sp \forall n \geq N \sp |f_n(c)| \leq 1/m$
    Which converges to $0$. Thus $f_n \to 0$ at $c$. So the polynomials approximating $f$ may be 
    taken to vanish at $c$.

\end{proof}


\problem{3} 7.5.5 Problem 14

\begin{enumerate}[(a)]
    \item For $c_m = \int_{-1} ^1 (1-x^2)^m dx$ obtain the identity $c_m = c_{m-1} - (1/2m)c_m$ 
    by integration by parts.
    \item Show that 
    $$c_m = 2 \frac{2 \cdot 4 \cdot 6 \cdot ... \cdot 2m}{3 \cdot 5 \cdot 7 \cdot ... \cdot 2m +1} = \frac{2(2^mm!)^2}{(2m +1)!}$$
    
\end{enumerate}

\hrule

\begin{enumerate}[(a)]

    \item We know the integration by parts formula $\int u dv = uv - \int v du$ from calculus
    Let $u = (1-x^2)^m$, and $v = x$. Then $dv = dx$ and $du = m \cdot (1 - x^2)^{m-1} (-2x) = -2mx(1 - x^2)^{m-1}$
    \begin{align*}
        c_m &= \int_{-1} ^1 (1-x^2)^m dx\\
        &= \Eval{x(1-x^2)^m}{-1}{1} - \int_{-1} ^1 x \cdot -2mx(1 - x^2)^{m-1} dx \\
        &= 0 - \int_{-1} ^1 -2m \cdot x^2(1 - x^2)^{m-1} dx \\
        &= \int_{-1} ^1 2m \cdot (x^2 -1 +1)(1 - x^2)^{m-1} dx \\
        &= \int_{-1} ^1 2m \cdot (-(1- x^2) +1)(1 - x^2)^{m-1} dx \\
        &= \int_{-1} ^1 2m \cdot -(1- x^2)(1 - x^2)^{m-1} + \int_{-1} ^1 2m \cdot (1 - x^2)^{m-1}dx \\
        &= \int_{-1} ^1 2m \cdot (1 - x^2)^{m-1}dx - \int_{-1} ^1 2m \cdot (1 - x^2)^{m} dx\\
        c_m &= 2m \cdot c_{m-1} - 2m \cdot c_m\\
        c_m &= 2m \cdot (c_{m-1} - c_m)\\
        1/2m \cdot c_m &= (c_{m-1} - c_m)\\
        c_m &= c_{m-1} - 1/2m \cdot c_m
    \end{align*}

    \item 

    \begin{proof} By induction
        
        Base case: we show $c_1 = \frac{2(2^1 1!)^2}{(2+1)!} = 8/6 = 4/3$
        We then verify that \\
        $c_1 = 4/3$. $\int_{-1} ^1 (1 - x^2) dx = \int_{-1} ^1 dx - \int_{-1} ^1 x^2 dx = 2 - \Eval{1/3 x^3}{-1}{1} = 2 - 2/3 = 4/3$

        Then the base case holds. So assume that $c_{m-1} = \frac{2(2^{m-1}(m-1)!)^2}{(2(m-1) +1)!}$
        
        We know from above that $c_{m-1} = c_m (1 + 1/2m)$ So then
        \begin{align*}
            c_m (1 + 1/2m) &= c_{m-1}\\
            c_m (1 + 1/2m) &= \frac{2(2^{m-1}(m-1)!)^2}{(2m -1)!}\\
            c_m &= \frac{2(2^{m-1}(m-1)!)^2}{(2m -1)! (1 + 1/2m)}\\
            c_m &= \frac{2(2^{m-1}(m-1)!)^2}{(2m -1)! + (2m -1)!/2m}\\
            c_m &= \frac{2(2m)(2^{m-1}(m-1)!)^2}{2m(2m -1)! + (2m -1)!}\\
            c_m &= \frac{2(2m)(2m)(2^{m-1}(m-1)!)^2}{(2m -1)!(2m + 1)(2m)}\\
            c_m &= 2\frac{(2m)(2^{m-1}(m-1)!)(2m)(2^{m-1}(m-1)!)}{(2m + 1)!}\\
            c_m &= 2\frac{(2^{m}m!)(2^{m}m!)}{(2m + 1)!}\\
            c_m &= 2\frac{(2^{m}m!)^2}{(2m + 1)!}\\
        \end{align*}

        Then the inductive step holds. So the claim is true by mathematical induction.

    \end{proof}

\end{enumerate}

\end{document}