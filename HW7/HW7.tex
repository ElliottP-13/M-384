\documentclass[11pt]{article}
\usepackage{../EllioStyle}

\title{Homework 7}
\author{Elliott Pryor}
\date{29 March 2021}

\rhead{Homework 7}

\begin{document}
\maketitle

\problem{1} 7.6.3 Problem 2

If $|f_n(x) - f_n(y)| \leq M |x - y|^\alpha$ for some fixed $M$ and $\alpha > 0$
and all $x,y$ in a compact interval. Show that $\{f_n\}$ is uniformly
equicontinuous

\hrule

\begin{proof}
    
    By definition of uniform equicontinuity we need to show:
    $\forall 1/m \sp \exists 1/n \sp st \sp |x - y| < 1/n \implies |f_k(x) - f_k(y)| < 1/m \sp \forall k$
    We have from the definition of $f_n$ that $|f_n(x) - f_n(y)| \leq M |x - y|^\alpha$,
    So $|f_k(x) - f_k(y)| \leq M |x - y|^\alpha \leq M (1/n)^\alpha$ which we need to be $< 1/m$.
    So we choose $1/n < \sqrt[\alpha]{\frac{1}{Mm}}$ since $M, \alpha$ are defined and $\alpha > 0$
    we can compute this for any $1/m$ as required.
\end{proof}



\problem{2} 7.6.3 Problem 5

Give an example of a sequence that is uniformly equicontinuous 
but not uniformly bounded

\hrule

$f_n = x + n$. Is uniformly equicontinuous since can choose $n = m$
Since we have $|f_n(x) - f_n(y)| = |x - y|$. 
But is not uniformly bounded
since $f_n(x) \to \infty$ as $n \to \infty$ for any $x$.




\problem{3} 7.6.3 Problem 6

Prove that the family of all polynomials of degree $\leq N$ with
coefficients in $[-1,1]$ is uniformly bounded and uniformly
equicontinuous on any compact interval. 

\textbf{Hint: }
Let the compact interval be $[a, b]$, and each polynomial in the family has the form

$p(x) = c_0 + c_1 x + c_2 x^2 + \dots + c_N x^N \sp x \in [a,b], \sp c_i \in [-1, 1], \sp 0 \leq i \leq N$

To prove uniform equicontinuity, show that the derivatives of all polynomials in the 
family are uniformly bounded.

\hrule

\begin{proof}
    
Consider the derivatives of the family of functions:
$p'(x) = c_1 + 2 c_2 x + 3 c_3 x^2 + \dots + N c_{N} x ^{N-1}$
Then let $g$ be the polynomial with all coefficients = 1:
$g(x) = 1 + x + x^2 + \dots + x^N$, and $g'(x) = 1 + 2x + 3x^2 + \dots + N x^{N-1}$.
Let $M = \sup{x \in [a,b]} g(x)$ which exists since $g$ is continuous. Then
clearly term by term $p \leq g$, so any $p$ is bounded by $M$ thus is uniformly bounded.

Then clearly outside of $(-1,1)$ $p \leq g$ and $p' \leq g'$ for any $p$ in the family.
We can easily see this by comparing each term in $|p'| = |i \cdot c_i \cdot x^{i-1}| \leq i \cdot 1 \cdot x^{i-1} = g'$.
Let $f = \sup_{x \in [a,b]} g'(x)$
which we know exists since $g'$ is continuous (thus it is bounded).

And inside $(-1, 1)$, $p'(x) \leq \sum_{i = 1} ^N i$ since for any term
$i \cdot c_i x^{i-1} \leq i \cdot x^{i-1} \leq i$ inside $x \in (-1,1)$.
We call this summation $e = \sum_{i = 1} ^N i$. 

Then let $M = \max (e, f)$. So then clearly $p'$ is uniformly bounded by $M$.

Then copied from the proof of Corollary 7.6.1: by MVT
$\forall x, y \in [a,b] \sp \forall k \sp |f_k(x) - f_k(y)| = |f'_k(z)(x-y)| \leq |f'_k(z)||x-y| \leq M|x-y| < 1/m$
by choosing $1/n > \frac{1}{mM}$.
Thus all $p$ are equicontinuous
\end{proof}


\problem{4} 7.6.3 Problem 9

Give an example of a uniformly bounded and uniformly equicontinuous
sequence of functions on the whole line that does not have any uniformly
convergent subsequences.

\textbf{Hint: } Consider the following sequence of functions on $\reals$

$$f_n(x) = \begin{cases}
    0 & x \leq n-1\\
    x - (n-1) & n-1 < x \leq n\\
    1 & x > n
\end{cases}$$

\hrule

\begin{proof}

    Let $\{f_n\}$ be:

    $$f_n(x) = \begin{cases}
        0 & x \leq n-1\\
        x - (n-1) & n-1 < x \leq n\\
        1 & x > n
    \end{cases}$$

    Clearly $f_n$ are uniformly bounded by $M = 1$ for all $x \in \reals$.
    Then we show that it is equicontinuous:
    $\forall 1/m \sp \exists 1/n \sp st \sp |x - y| < 1/n \implies |f_k(x) - f_k(y)| < 1/m \sp \forall k$
    If $x,y \notin (n-1, n]$ then $|f_k(x) - f_k(y)| = 0$.
    So then we only need to consider the middle portion. This is just
    a line of slope 1 with x-intercept at $n-1$, so it slants from 0, to 1 over interval $[n-1, n]$.
    If $x,y \in (n-1, n]$ then choose $1/n = 1/m$ since: 
    $|f_k(x) - f_k(y)| = |x - (n-1) - y + (n-1)| = |x-y|$
    if only $x$ or $y$ are in $(n-1, n]$ then also choose $1/n = 1/m$
    since $(x,y)$ spans the 'corner' of the function, so $|f_k(x) - f_k(y)| \leq |x-y|$.

    Consider any subsequence of $f_n$. Then suppose it meets the uniform convergence criteria.
    So then the Cauchy criteria is met:
    $\forall 1/m \sp \exists N \sp st \sp \forall j,k \geq N \sp \forall x \in \reals |f_j(x) - f_k(x)| < 1/m$
    Consider $k \geq j + 4$ and $x = j + 2$. Then $f_j(j + 2) = 1$
    and $f_k(j + 2) = 0$ Thus $f_j(x) - f_k(x) = 1 > 1/m$ for $m > 1$.
    We can always do this for any $j,k$. So it violates the Cauchy criterion
    for any subsequence, thus there are no uniformly convergent subsequences.

\end{proof}



\end{document}