\documentclass[11pt]{article}
\usepackage{../EllioStyle}

\title{Homework 6}
\author{Elliott Pryor}
\date{19 March 2021}

\rhead{Homework 5}

\begin{document}
\maketitle

\problem{1} 7.5.5 Problem 7

If $f$ is $C^1$ on $[a,b]$ prove that there exists a cubic polynomial $P$
such that $f- P$ and its first derivative vanish at the endpoints of the interval.

\hrule

\begin{proof}
    
    
\end{proof}



\problem{2} 7.5.5 Problem 9 

If $f(c) = 0$ for some $c \in (a,b)$ prove that the polynomials approximating $f$ on 
$[a,b]$ may be taken to vanish at $c$.

\textbf{Hint: } Here $f(x)$ is a continuous function on $[a,b]$.
Assume $f_n(x)$ is the sequence of polynomials approximating $f(x)$ uniformly by WTA,
consider $g_n(x) = f_n(x) - f_n(c)$

\hrule


\begin{proof}
    
    Let $f_n$ be the sequence of polynomials approximating $f$ by the Wierstrass Approximation Theorem
    By WAT, we know that $f_n \to f$ uniformly. Thus
    $\forall 1/m \sp \exists N \sp st \sp \forall n \geq N \sp \forall x \in [a,b] \sp |f_n(x) - f(x)| \leq 1/m$.
    So we know that at $x = c$ we have $\forall 1/m \sp \exists N \sp st \sp \forall n \geq N \sp |f_n(c)| \leq 1/m$
    Which converges to $0$. Thus $f_n \to 0$ at $c$. So the polynomials approximating $f$ may be 
    taken to vanish at $c$.

\end{proof}


\problem{3} 7.5.5 Problem 14

\begin{enumerate}[(a)]
    \item For $c_m = \int_{-1} ^1 (1-x^2)^m dx$ obtain the identity $c_m = c_{m-1} - (1/2m)c_m$ 
    by integration by parts.
    \item Show that 
    $$c_m = 2 \frac{2 \cdot 4 \cdot 6 \cdot ... \cdot 2m}{3 \cdot 5 \cdot 7 \cdot ... \cdot 2m +1} = \frac{2(2^mm!)^2}{(2m +1)!}$$
    
\end{enumerate}

\hrule

\begin{enumerate}[(a)]

    \item We know the integration by parts formula $\int u dv = uv - \int v du$ from calculus
    Let $u = (1-x^2)^m$, and $v = x$. Then $dv = dx$ and $du = m \cdot (1 - x^2)^{m-1} (-2x) = -2mx(1 - x^2)^{m-1}$
    \begin{align*}
        c_m &= \int_{-1} ^1 (1-x^2)^m dx\\
        &= \Eval{x(1-x^2)^m}{-1}{1} - \int_{-1} ^1 x \cdot -2mx(1 - x^2)^{m-1} dx \\
        &= 0 - \int_{-1} ^1 -2m \cdot x^2(1 - x^2)^{m-1} dx \\
    \end{align*}

\end{enumerate}

\end{document}