\documentclass[11pt]{article}
\usepackage{../EllioStyle}

\title{Homework 8}
\author{Elliott Pryor}
\date{9 April 2021}

\rhead{Homework 9}

\begin{document}
\maketitle

\problem{1} 10.1.5 Problem 3

If $f$ is differentiable at $y$, show that $d_uf(y)$ is linear in $u$,
meaning $d_{(au + bv)}f(y) = a d_u(y) + b d_v f(y)$.

\textbf{Hint:} Apply Theorem 10.1.1

\hrule


\begin{proof}
    
    We know from theorem 10.1.1, since $f$ is differentiable at $y$, then
    $d_{(au + bv)}f(y) = df(y) (au + bv) = a \cdot df(y) u + b \cdot df(y)  v$
    since matrix multiplication is distributive. 
    We then recognize that $df(y) u$ has the form (from theorem 10.1.1) of $d_u(y)$,
    and similarly for $df(y) v$. So we have 
    $d_{(au + bv)}f(y) = df(y) (au + bv) = a \cdot df(y) u + b \cdot df(y)  v = a d_u(y) + b d_v f(y)$
\end{proof}





\problem{2} 10.1.5 Problem 10

Let $g : [a,b] \to \reals^n$ be differentiable. If $f : \reals^n \to \reals$ is differentiable,
what is the derivative $(d/dt)f(g(t))$

\textbf{Hint:} Use notation $g(t) = (g_1(t), \dots, g_n(t)), t \in [a,b]$
and $f(z) = f(z_1, \dots, z_n), z = (z_1, \dots z_n) \in \reals^n$ and Apply
the chain rule.

\hrule


\begin{proof}
    
    we know the Chain rule in general is: 
    $$\frac{\partial f}{\partial x_j} = \sum_{k = 1} ^n \frac{\partial f}{\partial z_k} \frac{\partial z_k}{\partial x_j}$$
    where $z_k = g_k(x_1, \dots x_n)$.

    In our case, we are looking for $x_j = t$, and $g(t) = (g_1(t), \dots, g_n(t)), t \in [a,b]$
    So $z_k = g_k(t)$.

    So we have 
    $$\frac{\partial f}{\partial t} = \sum_{k = 1} ^n \frac{\partial f}{\partial z_k} \frac{\partial g_k(t)}{\partial t}$$
    $\frac{\partial g_k(t)}{\partial t}$ is just a number (scalar), so we can write this as two separate sums:
    $\sum_{k = 1} ^n \frac{\partial f}{\partial z_k} * \sum_{k = 1} ^n \frac{\partial g_k(t)}{\partial t}$.
    Then clearly this is just the sum of all partial derivatives of each function, which is the differential $df, dg$.
    We also note that this $df = \nabla f$ since $f: \reals ^n \to \reals$.
    So we have:

    $$\frac{\partial f}{\partial t} = df \, dg/dt = \nabla f * dg/dt$$

\end{proof}






\problem{3} 10.1.5 Problem 13

Compute $df$ of 
\begin{enumerate}[a.]
    \item $f: \reals^2 \to \reals, \sp f(x_1, x_2) = x_1 e^{x_2}$
    \item $f: \reals^3 \to \reals^2, \sp f(x_1, x_2, x_3) = (x_3, x_2)$
    \item $f: \reals^2 \to \reals^3, \sp f(x_1, x_2) = (x_1, x_2, x_1\cdot x_2)$
\end{enumerate}

\hrule


We know that for the differential matrix $df$ we know that 
$df_{k,j} = \frac{\partial f_k}{\partial x_j}$

\begin{enumerate}[a.]
    \item $f: \reals^2 \to \reals, \sp f(x_1, x_2) = x_1 e^{x_2}$
    There is just one, $f_k$ and there are $x_1, x_2$. So we have that 
    $$df = \begin{bmatrix}
        \frac{\partial f}{\partial x_1} & \frac{\partial f}{\partial x_2}
    \end{bmatrix} = 
    \begin{bmatrix}
        e^{x_2} & x_1 e^{x_2}
    \end{bmatrix}$$

    \item $f: \reals^3 \to \reals^2, \sp f(x_1, x_2, x_3) = (x_3, x_2)$
    
    $$df = \begin{bmatrix}
        \frac{\partial f_1}{\partial x_1} & \frac{\partial f_1}{\partial x_2} & \frac{\partial f_1}{\partial x_3}\\
        \frac{\partial f_2}{\partial x_1} & \frac{\partial f_2}{\partial x_2} & \frac{\partial f_2}{\partial x_3}
    \end{bmatrix} = 
    \begin{bmatrix}
        0 & 0 & 1 \\
        0 & 1 & 0 \\
    \end{bmatrix}$$

    \item $f: \reals^2 \to \reals^3, \sp f(x_1, x_2) = (x_1, x_2, x_1\cdot x_2)$
    
    $$df = \begin{bmatrix}
        \frac{\partial f_1}{\partial x_1} & \frac{\partial f_1}{\partial x_2}\\
        \frac{\partial f_2}{\partial x_1} & \frac{\partial f_2}{\partial x_2}\\
        \frac{\partial f_3}{\partial x_1} & \frac{\partial f_3}{\partial x_2}
    \end{bmatrix} =
    \begin{bmatrix}
        1, 0 \\
        0, 1 \\
        x_2, x_1
    \end{bmatrix}$$

\end{enumerate}




\problem{4} 10.1.5 Problem 15

If $f: D \to \reals$ is $C^1$ with $D \subseteq \reals^n$ and $D$ contains the line segment
joining $x$ and $y$, show that $f(y) = f(x) +  \nabla f(z) \cdot (y - x)$ for some
point $z$ on the line segment. Explain why this is an $n$-dimensional
analog of the mean value theorem


\textbf{Hint:} Define function $g : [0, 1] \to \reals^n$ by $g(t) = x + t(y - x)$
and consider the composition function 
$$h(t) = (f \circ g)(t) = f(g(t)) : \reals \to \reals$$
Apply Mean Value Theorem to $h(t)$ for $h(1)- h(0)$ and use the chain rule (formula derived
in problem 10 above) to calculate $h'$
\hrule


\begin{proof}
    
    Let $g : [0, 1] \to \reals^n$ by $g(t) = x + t(y - x)$, and let $h$ be the composition
    function: $h(t) = (f \circ g)(t) = f(g(t)) : \reals \to \reals$.
    By our definition of $g$: $g(0) = x, \; g(1) = y$ and $g(t)$ is on the straight line-segment
    joining $x$ to $y$. Thus the image of $g \in D$. Then $f \in C^1$ so $h$ is differentiable.

    From problem 10: we know that $d/dt (h) = \nabla f * dg/dt$, and we know that $dg/dt = (y-x)$ by definition of $g$.
    Then by Mean value Theorem we know that
    there exists a $z$ such that: $(1-0) * h'(z) = h(1) - h(0) = f(g(1)) - f(g(0)) = f(y) - f(x)$
    Since we know $h' = dh/dt = \nabla f(z) * (y - x)$ we have:
    $f(y) - f(x) = \nabla f(z)(y - x)$ we simply re-arrange this to get:
    $$f(y) = f(x) + \nabla f(z)(y-x)$$
\end{proof}

We see that this is the mean value theorem by further re-arranging: 
$\nabla f(z) = df(z)$ since $f: \reals ^n \to \reals$

$$df(z) = \frac{f(y) - f(x)}{y-x}$$

which exactly matches the mean value theorem.




\end{document}