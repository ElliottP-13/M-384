\documentclass[11pt]{article}
\usepackage{../EllioStyle}

\title{Homework 9}
\author{Elliott Pryor}
\date{20 April 2021}

\rhead{Homework 9}

\begin{document}
\maketitle

\problem{1} 

Let $f: \reals ^2 \to \reals$ be defined by:

$$f(x,y) = \begin{cases}
    \frac{xy(x^2-y^2)}{x^2 + y^2} & x,y \neq 0,0 \\
    0 & x, y = 0, 0
\end{cases}$$

\begin{enumerate}[(a)]
    \item Show that $\partial f / \partial x, \partial f / \partial y$ exist for all $(x, y) \in \reals ^2$
    
    \textbf{Hint: } for $(x,y) \neq (0,0)$ calculate directly by formula. For $(x, y) = (0,0)$ calculate by its definition:

    $$\frac{\partial f}{\partial x}(0,0) = \lim_{x \to 0} \frac{f(x, 0) - f(0,0)}{x-0}$$

    \item Show that both $\frac{\partial ^2 f(0,0)}{\partial x \partial y}$ and $\frac{\partial ^2 f(0,0)}{\partial y \partial x}$
    exist, but $\frac{\partial ^2 f(0,0)}{\partial x \partial y} \neq \frac{\partial ^2 f(0,0)}{\partial y \partial x}$

    \textbf{Hint:} note
    $$\frac{\partial ^2 f(0,0)}{\partial x \partial y} = 
    \frac{\partial}{\partial x} \left( \frac{\partial f}{\partial y} \right) = 
    \lim_{x \to 0} \frac{\frac{\partial f(x,0)}{\partial y} - \frac{\partial f(0,0)}{\partial y}}{x - 0}$$

    Where $\frac{\partial f(x,0)}{\partial y}, \frac{\partial f(0,0)}{\partial y}$ are calculated in part (a)

\end{enumerate}
\hrule

\begin{enumerate}[(a)]
    \item 
    \begin{proof}
        We start by considering $x, y \neq 0$. Then we directly compute $\partial f / \partial x$
        We have $\frac{xy(x^2-y^2)}{x^2 + y^2} = \frac{yx^3-xy^3}{x^2 + y^2}$

        We use the quotient rule to get: 
        $\partial f / \partial x = \frac{(3yx^2 - y^3)(x^2 + y^2)  - (yx^3-xy^3)2x}{(x^2 + y^2)^2} = \frac{(3yx^2 - y^3)(x^2 + y^2) - 2yx^4-2x^2y^3}{(x^2 + y^2)^2}$

        Similarly, we solve: $\partial f / \partial y = \frac{(x^3 - 3xy^2)(x^2+ y^2) - (yx^3-xy^3)2y}{(x^2 + y^2)^2} = \frac{(x^3 - 3xy^2)(x^2 + y^2) - 2y^2x^3-2xy^4}{(x^2 + y^2)^2}$
        
        Then at $x, y = 0$ we use the definition. 
        So for $y = 0$
        \begin{align*}
            \partial f / \partial x &= \lim_{x \to 0} \frac{f(x,0) - f(0,0)}{x - 0}\\
            &= \lim_{x \to 0} \frac{\frac{0 \cdot x^3-x \cdot 0^3}{x^2 + 0^2}}{x} \\
            &= \lim_{x \to 0} 0/x^2 \cdot 1/x \\
            &= \lim_{x \to 0} 0/x^3 \\
            &=  \lim_{x \to 0} 0/1 = 0 \\
        \end{align*}
        
        Where the last step follows from applying L'hopitals rule 3 times.

        We repeat the same process for $y$
        \begin{align*}
            \partial f / \partial y &= \lim_{y \to 0} \frac{f(0,y) - f(0,0)}{y - 0}\\
            &= \lim_{y \to 0} \frac{\frac{y \cdot 0^3- 0 \cdot y^3}{0^2 + y^2}}{y} \\
            &= \lim_{y \to 0} 0/y^2 \cdot 1/y \\
            &= \lim_{y \to 0} 0/y^3 \\
            &=  \lim_{y \to 0} 0/1 = 0 \\
        \end{align*}

        Thus the partial derivatives exist everywhere.

    \end{proof}





    
    \item We compute the second partial derivatives:
    \begin{proof}
        
        \begin{align*}
            \frac{\partial ^2 f(0,0)}{\partial x \partial y} = \frac{\partial}{\partial x} \left( \frac{\partial f}{\partial y} \right) &= 
                \lim_{x \to 0} \frac{\frac{\partial f(x,0)}{\partial y} - \frac{\partial f(0,0)}{\partial y}}{x - 0}\\
            &= \lim_{x \to 0} \frac{\frac{(x^3 - 3x \cdot 0^2)(x^2 + 0^2) - 2 \cdot 0^2x^3-2x \cdot 0^4}{(x^2 + 0^2)^2} - 0}{x} \\
            &= \lim_{x \to 0} \frac{\frac{x^5}{x^4} - 0}{x} \\
            &= 1\\
        \end{align*}

        \begin{align*}
            \frac{\partial ^2 f(0,0)}{\partial y \partial x} = \frac{\partial}{\partial y} \left( \frac{\partial f}{\partial x} \right) &= 
                \lim_{y \to 0} \frac{\frac{\partial f(0,y)}{\partial x} - \frac{\partial f(0,0)}{\partial x}}{y - 0}\\
            &= \lim_{y \to 0} \frac{\frac{(3y \cdot 0^2 - y^3)(0^2 + y^2) - 2y \cdot 0^4-2 \cdot 0^2y^3}{(0^2 + y^2)^2} - 0}{y}\\
            &= \lim_{y \to 0} \frac{\frac{- y^5}{y^4} - 0}{y}\\
            &= -1
        \end{align*}

    \end{proof}
\end{enumerate}




\problem{2}

For any $x = (x_1, x_2, \dots, x_n) \in \reals ^n$ and any multi-index $\alpha = (\alpha_1, \dots, \alpha_n)$ prove that 
$$|x^\alpha| \leq |x|^{|\alpha|}$$
where $x^\alpha = x_1^{\alpha_1} \dots x_n^{\alpha_n}$, $|x| = \sqrt{x_1^2 + \dots + x_n^2}$, $|\alpha| = \alpha_1 + \dots + \alpha_n$

\hrule

\end{document}