\documentclass[11pt]{article}
\usepackage{../EllioStyle}

\title{Homework 5}
\author{Elliott Pryor}
\date{08 March 2021}

\rhead{Homework 5}

\begin{document}
\maketitle

\problem{1} 7.4.5 Problem 2

If $f$ is analytic in a neighborhood of $x_0$ and $f(x_0) = 0$, 
show that $f(x)/(x - x_0)$ is analytic in the same neighborhood.

\textbf{Hint:} Write $f(x)$ as a power series expanded at $x_0$ and pay attention to what is $a_0$

\hrule

\begin{proof}
    
    Since $f$ is analytic around $x_0$ we can write it as the power series: 
    $f(x) = \sum{n=0} ^\infty a_n (x - x_0)^n = a_0 + a_1 (x - x_0) + a_2 (x - x_0)^2$.
    Then $f(x_0) = a_0 = 0$ by definition of $f$.  

    So then we can write it as $\sum{n=1} ^\infty a_n (x - x_0)^n = a_1 (x - x_0) + a_2 (x - x_0)^2$.
    We can divide by $x - x_0$: $\sum{n=1} ^\infty a_n (x - x_0)^{n-1}$.
    We then let $n' = n-1$ so we have\\
    $\frac{f(x)}{x-x_0} = \sum_{n'=0} ^\infty a_{n'+1} (x - x_0)^{n'}$
    which is a power series expansion about $x_0$.

    Then in order to show that they are analytic we just need to show that they have the same radius of convergence.
    Then by analytic continuation they are analytic on the same neighborhood $(x_0 - R, x_0 + R)$.
    We know that $\limsup_{n \to \infty} |a_n|^{1/n} = 1/R$ by definition.
    We need to show that $\limsup_{n \to \infty} |a_{n+1}|^{1/n} = 1/R$

    \begin{align*}
        \limsup_{n \to \infty} |a_{n+1}|^{1/n} \\
        \limsup_{n \to \infty} |a_{n+1}|^{\frac{1}{n+1} \cdot \frac{n+1}{n}}\\
        \limsup_{n \to \infty} \left(|a_{n+1}|^{\frac{1}{n+1}}\right)^{\frac{n+1}{n}}\\
        \limsup_{n \to \infty} (1/R)^{\frac{n+1}{n}}\\
        1/R
    \end{align*}
    where the last step follows from $\lim_{n \to \infty} \frac{n+1}{n} = 1$
\end{proof}






\problem{2} 7.4.5 Problem 6

Prove that if $f(x)$ is analytic on $(a,b)$, then $F(x) = \int_c ^x f(t)dt$ 
is also analytic on $(a,b)$, where $c$ is any point in $(a,b)$

\textbf{Hint:} You need to prove that for any $x_0 \in (a,b)$, $F(x)$ has
a power series expansion about $x_0$ for $x$ close to $x_0$.
Now pick any fixed $x_0 \in (a,b)$ and write $F(x)$ as:

$$F(x) = \int_c ^x f(t) dt = \int_c ^{x_0} f(t) dt + \int_{x_0} ^x f(t) dt = C_0 + \int_{x_0} ^x f(t)dt$$

\hrule

\begin{proof}
    
    We need to show that $F(x)$ has a power series expansion about any point $x_0 \in (a,b)$.
    So we pick any fixed $x_0 \in (a,b)$.
    Then: $F(x) = \int_c ^x f(t) dt = \int_c ^{x_0} f(t) dt + \int_{x_0} ^x f(t) dt = C_0 + \int_{x_0} ^x f(t)dt$
    Then we can write $f(t) = \sum_{n=0} ^\infty a_n (x-x_0)^n$
    since it is analytic on $(a,b)$ and $x_0 \in (a,b)$, so $f$ has a power series expansion about $x_0$.

    So we have:

    \begin{align*}
        F(x) &= C_0 + \int_{x_0} ^x f(t)dt\\
        &= C_0 + \int_{x_0} ^x \sum_{n=0} ^\infty a_n (t-x_0)^n dt\\
        &= C_0 + \sum_{n=0} ^\infty a_n \int_{x_0} ^x  (t-x_0)^n dt \quad \quad \text{By linearity of integral}\\
        &= C_0 + \sum_{n=0} ^\infty  a_n \left(  1/(n+1) (t-x_0)^{n+1} \right| _{x_0}^x\\
        &= C_0 + \sum_{n=0} ^\infty  a_n \left(  1/(n+1) (x-x_0)^{n+1} - 1/(n+1) (x_0-x_0)^{n+1}\right)\\
        &= C_0 + \sum_{n=0} ^\infty  a_n/(n+1) (x-x_0)^{n+1}\\
    \end{align*}

    Let $b_0 = C_0$ and $b_n = \frac{a_{n-1}}{n+1}$ for $n \geq 1$. Then we have:
    $F(x) = \sum_{n=0} ^\infty b_n (x-x_0)^{n+1}$. 



\end{proof}








\problem{3} 7.4.5 Problem 7 a.

Compute the power-series expansion of $f(x) = \frac{x^2}{1-x^2}$ about $x=0$

\hrule







\problem{4}

Compute the radius of convergence of the following power series:

\begin{enumerate}[a.]
    \item $\sum (n^4/n!) \cdot x^n$
    \item $\sum \sqrt{n} x^n$
    \item $(n^2 2^n) x^n$
\end{enumerate}

\hrule




\end{document}