\documentclass[11pt]{article}
\usepackage{../EllioStyle}

\title{Homework 4}
\author{Elliott Pryor}
\date{24 Feb 2021}

\rhead{Homework 4}

\begin{document}
\maketitle

\problem{1} 7.3.4 Problem 2

Suppose $f_n \to f$ and all functions $f_n$ satisfy the Lipschitz condition
$|f_n(x) - f_n(y)| \leq M |x-y|$ for some constant $M$, independent of $n$.
Prove that $f$ also satisfies the same Lipschitz condition

\hrule


\begin{proof}
    We know that by the definition of convergence that $\forall x_0$ the sequence of numbers
    $f_1(x_0), f_2(x_0), ...$ converges to $f(x_0)$. Then we have that 
    $\forall 1/n \sp \exists m \sp st \sp \forall k > m \sp |f_k(x_0) - f(x_0) | < 1/n$.

    We want to show that $|f(x) - f(y)| \leq M |x-y|$. So we expand,

    \begin{align*}
        |f(x) - f(y)| &= |f(x) - f_k(x) + f_k(x) - f(y) + f_k(y) - f_k(y)| \\
        &\leq |f(x) - f_k(x)|  + |f_k(x) - f_k(y)| + |f(y) - f_k(y)|\\
        &\leq |f(x) - f_k(x)|  + M|x - y| + |f(y) - f_k(y)|\\
    \end{align*}

    We can then set $k$ such that $|f(x) - f_k(x)| \leq M |x - y|$, and same for $y$ (we choose the maximum $k$).
    We know we can do this by the convergence of $f_k$. 

    Then we have $|f(x) - f(y)| \leq  M|x - y| +  M|x - y| +  M|x - y| = 3 M|x - y|$.
    Thus $f$ satisfies the Lipschitz condition.
\end{proof}







\problem{2} 7.3.4 Problem 5

If $\lim_{n \to \infty} f_n = f$ and the functions $f_n$ are all monotone increasing, must $f$ be monotone increasing?
What happens if $f_n$ are all strictly increasing?

\hrule

Yes, if $f_n$ are all monotone increasing, then $f$ is monotone increasing.

\begin{proof}
    By contradiction.

    Suppose not, suppose $\exists y > x, \sp st \sp f(y) < f(x)$ (f decreases).
    Then the sequence of numbers $f_1(y), f_2(y), ...$ converges to $f(y)$
    and $f_1(x), f_2(x), ...$ converges to $f(x)$. But since each $f_k(x)$ is monotone increasing.
    We have $f_1(x) \leq f_1(y), f_2(x) \leq f_2(y), ...$
    Since non-strict inequality is preserved under the limit: we then have $f(x) \leq f(y)$
    a contradiction.
\end{proof}

If $f_n$ are strictly increasing, then $f$ is not necessarily strictly increasing.
(Just monotone increasing).

Consider the common domain $\mathbb{D} = [2, 10]$. Let $f_n(x) = 1 - 1/nx$.
Then clearly $f_n$ are strictly increasing since if $x_1 > x_0$, then $1/x_1 < 1/x_0$.
And $f_n$ converge to $f(x) = 1$. Which is a constant function (by definition still monotone increasing).
But is clearly not strictly increasing.





\problem{3} 7.3.4 Problem 6

Give an example of a sequence of continuous functions converging pointwise to a function
with a discontinuity of the second kind.

\textbf{Hint:} Consider the common domain $\mathbb{D} = [0,1]$ and 
$$f_n (x) = \begin{cases}
    nx & 0 \leq x \leq 1/n \\
    1 & 1/n \leq x \leq 1 \\
\end{cases}$$

Find another function $g(x)$ which has a discontinuity of the second kind on $\mathbb{D}$ and define 
$g_n(x) = f_n(x) \cdot g(x)$
You need to prove that $g_n$ are continuous on $\mathbb{D}$ and converges pointwise to a function with a 
discontinuity of the second kind.

\hrule


So we pick $g(x)$ to be the zigzag function. $g(1) = 1, g(1/2) = 0, g(1/4) = 1, ... $ with linear line segments
connecting the points. We know that $g(x)$ has a discontinuity of the second kind at $x= 0$.

Let $$f_n (x) = \begin{cases}
    nx & 0 \leq x \leq 1/n \\
    1 & 1/n \leq x \leq 1 \\
\end{cases}$$

Then $g_n(x) = f_n(x) \cdot g(x)$. We show that $g_n$ is continuous.

Clearly, for $x > 1/n$ $g_n$ is continuous since $f_n(x) = 1$, so $g_n(x) = g(x)$ which we know to be continuous.
All the line segments are always continuous of $g$, so we just have to show that the discontinuity of second kind
at $x = 0$ does not occur.

By the squeeze theorem, $g_n \leq f_n$ for all $x < 1/n$ (this holds for larger than $1/n$ as well).
Then $\lim_{x\to 0} f_n(x) = 0$. So then using squeeze theorem $\lim_{x\to 0} g_n(x) = 0$.
Thus there is not a discontinuity of the second kind. There could be a jump discontinuity, so we check
$g_n(0) = f_n(0) \cdot g(0) = 0 \cdot g(0) = 0$ therefore, $g_n$ is continuous.

So next we show that $g_n$ converges pointwise to $g$. By the construction of $f_n$, we have 
that for $x > 1/n, \sp g_n(x) = g(x)$. So for any $x$, we choose $n$ large enough such that $1/n < x$, thus $g_n(x) = g(x)$.
So it converges pointwise to $g$. Since by choosing $n$ large enough $g_n = g$. 

Thus we found a sequence of continuous functions that converge pointwise to a function with a discontinuity of second kind.




\problem{4} 7.3.4 Problem 7

If $|f_n(x)| \leq a_n$ for all $x$, and $\sum_{n = 1}^ \infty a_n$ converges,
prove that $\sum_{n = 1}^ \infty f_n(x)$ converges uniformly.

\textbf{Hint:}
The series $\sum_{n = 1}^ \infty f_n(x)$ converges uniformly is the equivalent to that the sequence
of partial sum functions $F_n(x) = \sum_{k = 1} ^n f_k(x)$ converges uniformly. 
Then prove $F_n$ satisfies the Cauchy criterion for uniform convergence (Theorem 7.3.1).

\hrule

\begin{proof}
    We define the function $F_n(x) = \sum_{k = 1} ^ n f_k(x)$ to be the function of the partial sums. 
    We show that this converges uniformly.
    
    So, we show that $F_n$ satisfies the Cauchy Criterion. 

    $\forall 1/m \sp \exists N \sp st \sp \forall q,p \geq N \sp \forall x \sp |F_q(x) - F_p(x)| \leq 1/m$.

    Without loss of generality, assume $q \geq p$. Then 
    $|F_q(x) - F_p(x)| = \left| \sum_{i = p} ^q f_i(x) \right| \leq \sum_{i = p} ^q | f_i(x) | \leq \sum_{i = p} ^q a_i$

    Then, by the Cauchy Criterion for Convergence of Series (Theorem 7.2.1):
    $\forall 1/n \sp \exists m \sp st \sp \forall q \geq p \geq m \sp \left| \sum_{i = p} ^q a_i \right| < 1/n$.
    We observe that $a_n \geq 0$, so we can drop the absolute values signs $\sum_{i = p} ^q a_i < 1/n$.

    Thus, we have that $\forall 1/n \sp \exists m \sp st \sp \forall q \geq p \geq m \sp |F_q(x) - F_p(x)| \leq \sum_{i = p} ^q a_i  < 1/n$

\end{proof}






\end{document}