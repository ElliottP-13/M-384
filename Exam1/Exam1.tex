\documentclass[11pt]{article}
\usepackage{../EllioStyle}

\title{Exam 1}
\author{Elliott Pryor}
\date{03 March 2021}

\rhead{Exam 1}

\begin{document}
\maketitle

\problem{1}

True or False

\begin{enumerate}[(a)]
    \item if $f(x)$ on a finite interval $[a,b]$ is Riemann integrable, 
    then $f(x)$ can only have jump discontinuities.

    False. - Hole
    
    \item If $|f(x)|$ is Riemann integrable on $[a,b]$ 
    then $f(x)$ is also Riemann integrable on $[a,b]$.

    False

    \item if $\sum_{k = 1} ^ \infty a_k$ converges conditionally and 
    $\sum_{k = 1} ^ \infty b_k$ converges absolutely, 
    then $\sum_{k = 1} ^ \infty a_k \cdot b_k$ converges absolutely

    True

    \item If $\sum_{k = 1} ^ \infty a_k$ converges conditionally, 
    then the sum of all positive terms of this series diverges

    True

    \item Let $f_n$ be a sequence of $C^1$ functions defined on $(a,b)$
    if $f'_n (x)$ converges uniformly to $g(x)$ on $(a,b)$. 
    Then there exists a $C^1$ function $f$ on $(a,b)$ such that $f_n(x)$ converge uniformly to $f(x)$
    and $f'(x) = g(x)$

    True

\end{enumerate}


\problem{2}

Give the statement (using quantifiers) that a sequence of functions $f_n(x)$ 
on a common domain $\mathbb{D}$ does \textbf{NOT} converge uniformly to a function
$f(x)$ on $\mathbb{D}$

\hrule

a sequence of functions $f_n(x)$ 
on a common domain $\mathbb{D}$ does \textbf{NOT} converge uniformly to a function
$f(x)$ on $\mathbb{D}$ if: 

$$\exists 1/m \sp st \sp \forall N  \sp \exists x \in \mathbb{D}, \sp \exists n \geq N \sp st \sp |f_n(x) - f(x)| \geq 1/m$$






\problem{3}

Compute the radius of convergence of the power series 
$$\sum_{n=0} ^ \infty \frac{4n^5 - 7n + 2}{3^n} x^n$$

\hrule

Here we have: $a_n = \frac{4n^5 - 7n + 2}{3^n}$
We know the radius of convergence is given by: 
$\frac{1}{R} = \limsup_{n \to \infty} \sup \sqrt[n]{|a_n|}$

\begin{align*}
    \frac{1}{R} &= \limsup_{n \to \infty} \sqrt[n]{|a_n|}\\
    &= \limsup_{n \to \infty} \sqrt[n]{\left|\frac{4n^5 - 7n + 2}{3^n}\right|}\\
    &= \limsup_{n \to \infty} \left|\frac{\sqrt[n]{4n^5 - 7n + 2}}{\sqrt[n]{3^n}}\right|\\
    &= \frac{1}{3}
\end{align*}
(the numerator of the last step comes from the lemma that we used for 7.4.1
Or from the example we did showing radius of convergence of $a_n = p(n)/q(n)$ is 1, since $4n^5 -7n +2$ is a polynomial)

So $R = 3$




\problem{4}

Prove the first part of the linearity of Riemann Integral. 
Namely, if both $f(x)$ and $g(x)$ are Riemann integrable on $[a,b]$, 
then $f+g$ is Riemann integrable on $[a,b]$, and 

$$\int_a ^b (f + g)(x)dx = \int_a ^b f(x) dx + \int_a ^b g(x) dx$$

Here $(f + g)(x) = f(x) + g(x)$

\textbf{Hint:} Use part (e) of Theorem 6.2.1 and linear property of Cauchy sum


\hrule


\begin{proof}
    We take the Cauchy sum: $S(f+g, P)$ and show that it converges to $\int_a ^b f(x) dx + \int_a ^b g(x) dx$.
    
    Since $f,g$ are both bounded functions, then $f + g$ must also be bounded.

    By linearity of Cauchy sums, we have $S(f+g, P) = S(f, P) + S(g,P)$.
    By Theorem 6.2.1 part $e$, we have $S(f, P) \to \int_a ^b f(x) dx$ as the maximum
    interval length of P tends to zero. 

    So as $|P| \to 0$ we have that $S(f+g, P) = \int_a ^b (f + g)(x)dx$.
    We use the linearity of $S$ to find what the value of this is.
    $S(f+g, P) = S(f, P) + S(g,P) = \int_a ^b f(x) dx + \int_a ^b g(x) dx$
    by Theorem 6.2.1 and $f,g$ are Riemann Integrable.

\end{proof}







\problem{5}

Let $b_1, b_2, ...$ be a sequence of positive numbers convergent monotonically to zero:
$b_1 \geq b_2 \geq b_3 ...$ and $\lim_{n \to \infty} b_n = 0$.
If $|a_n| \leq b_n - b_{n+1}$ for all $n$. Prove:
$\sum_{n=1}^\infty a_n$ converges absolutely.

\hrule

\begin{proof}
    
    We show that $\sum_{n=1}^\infty a_n$ converges absolutely. So we show that:
    $\sum_{n=1}^\infty |a_n|$ converges by the Cauchy Criterion.

    $\forall 1/n \sp \exists m \sp st \sp \forall q \geq p \geq m \sp \sum_{k = p} ^q |a_k| < 1/n$.
    Since we know that $a_n$ is a sequence of positive numbers converging monotonically to zero:
    $b_n - b_{n+1} \geq 0$, so we can drop the absolute value symbol since all terms are positive.

    Then $\sum_{k = p} ^q a_k = b_p - b_{p+1} + b_{p+1} - b_{p+2} + b_{p+2} + ... - b_q + b_{q} - b_{q+1} = b_p - b_{q+1}$
    Since the sequence $b_n$ converges to zero, we can choose $m$ large enough that $b_m < 1/n$.
    Thus since $b_m \geq b_p \geq b_{q+1}$, we know that $b_p - b_{q+1} < 1/n$.



\end{proof}





\problem{6a}

Let $f_n(x) \to f(x)$ uniformly on a finite interval $[a,b]$ and all $f_n(x)$
are Riemann integrable on $[a,b]$. Define $F_n (x) = \int_a ^x f_n(t)dt$.
Prove that $F_n \to F$ uniformly on $[a,b]$ for some $F(x)$, 
and give the expression of the limit function $F(x)$.

\hrule


\begin{proof}

    We want to show that: 
    $\forall 1/m \sp \exists N \sp st \sp \forall x \in [a,b] \sp \forall n \geq N \sp |F_n(x) -F(x)| < 1/m$

    $|F_n(x) - F(x)| = \left|\int_a ^x f_n(t)dt - \int_a ^x f(x) dx \right|$
    Which then by linearity of the integral we have 
    $\left|\int_a ^x f_n(t)dt - \int_a ^x f(x) dx \right| = \left|\int_a ^x (f_n(t) - f(t)) dt \right| \leq \int_a ^x |f_n(t) - f(t)| dt $

    But since $f_n \to f$ uniformly: 
    $\exists N \sp st \sp \forall x \in [a,b] \sp \forall n \geq N \sp |f_n(x) -f(x)| < \frac{1}{m (b - a)}$.

    Then we have that $\int_a ^x |f_n(t) - f(t)| dt \leq \int_a ^x \frac{1}{m (b - a)} dt \leq \int_a ^b \frac{1}{m (b - a)} = 1/m$.
\end{proof}





\problem{6b}

Is the same true on the whole line? Namely, let $f_n(x) \to f(x)$ uniformly on the entire real line $\reals$
, and all $f_n(x)$ are Riemann integrable on any finite interval. Define $F_n(x) = \int_0 ^x f_n(t) dt$.
Is it always true that $F_n(x) \to F(x)$ uniformly on $\reals$ for some $F$?
Prove it if your answer is Yes, or give a counter example if your answer is No.

\hrule

No,

The problem comes in when we take $x \to \infty$.
If we take the sequence of constant functions $f_n (x) = (10 - 1/n)$ they clearly converge uniformly to $f(x) = 10$.
$F_n$ also converges uniformly to $F$ on any finite interval. But consider $\lim_{x \to \infty} |F_n(x) - F(x)|$.
$|F_n(x) - F(x)| = 1/n \cdot x$. We cannot choose an $N$ large enough such that for $n \geq N$, $x/n < 1/m \sp \forall x$.
Since we can always increase $x$ slightly to make $x/n > 1/m$.








\end{document}